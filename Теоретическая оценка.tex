\documentclass{report}

\usepackage[left=1cm,right=1cm,top=1cm,bottom=2cm]{geometry}

\usepackage[utf8]{inputenc}
\usepackage[russian]{babel}
\usepackage{amsmath}
\usepackage{amssymb}
\usepackage{wasysym}
\usepackage{listings}
\usepackage{sectsty}

\newcommand{\N}{\mathbb{N}}
\newcommand{\R}{\mathbb{R}}
\renewcommand{\phi}{\varphi}

\addto\captionsrussian{\renewcommand{\chaptername}{Задача}}
\chapternumberfont{\LARGE}
\chaptertitlefont{\Large}


\title{Лабораторная 1}
\author{Григоренко Павел, M3238}
\date{19-04-2023}


\begin{document}
\maketitle
\chapter{
Рассматривается генеральная совокупность из $(n+1)$ человек.
Человек, которого условимся называть прародителем,
пишет два письма случайно выбранным адресатам,
которые образую первое поколение.
Те в свою очередь делают то же самое,
в результате чего чего образуется второе поколение и тд.
Найти вероятность того, что прародитель не входит ни в одно из поколений с номерами $1,2,\dots,r$.
}

\chapter{
В квадрат наудачу брошены точки $A$, $B$.
Найти вероятность того, что круг, диаметром~которого является отрезок~$AB$,
целиком содержится в квадрате.
}

Перейдём от выбора двух концов диаметра к выбору центра и точки на окружности.
Правда центр, в отличие от концов диаметра, выбирается неравномерно.
Введём декартову систему координат так, что наш квадрат имеет длину стороны $1$,
а его стороны лежат на осях. Тогда $(x_0, y_0)=\left(\frac{x_a+x_b}{2}, \frac{y_a+y_b}{2}\right)$.
Так как $x_a$, $x_b$, $y_a$, $y_b$~--- независимые случайный величины,
равномерно выбранные на отрезке $[0;1]$, то $x_0$ и $y_0$ будут иметь ту же область значений,
но плотность вероятности будет
\[
 p_0(z)=\begin{cases}
    4z,\ z\in\left[0;\frac{1}{2}\right]\\
    4-4z,\ z\in\left(\frac{1}{2};1\right]
   \end{cases}
\] (думаю, этот факт достаточно очевиден).

Поймём, что, в принципе, не имеет значения, в какую четверть квадрата попал центр:
ситуация будет симметричной.
Так что пусть $x_0,y_0 \in \left[0;\frac{1}{2}\right]$.
Иначе просто расположим оси по-другому.

Зафиксируем $x_0$ и $y_0$.

Тогда $x_a \in [0; 2x_0]$, чтобы точка $x_b=2x_0-x_a\in[0;1]$.
Аналогично, $y_a\in[0;2y_0]$.

При этом $x_a$, $y_a$ равномерно распределены на своих интервалах.
В этом нетрудно убедиться, если рассмотреть их условные распределения:
\[
    p_{a|x_0=X_0}(x)=
    \frac{p_{x_a,x_0}(x, X_0)}{p_0(X_0)}=
    \frac{p_{x_a,x_b}(x, 2X_0-x)}{p_0(X_0)}=
    \frac{p_a(x) p_b(2X_0-x)}{p_0(X_0)}=
    \frac{1 \cdot 1}{p_0(X_0)}\text{~--- не зависит от $x$}
\]

При этом окружность будет содержаться в квадрате тогда и только тогда, когда
$\rho(C, A) \le \min(\rho(C, OX), \rho(C, OY))$, где $C=(x_0, y_0)$.

То есть, из всего прямоугольника возможных положений $А$,
нам подходит только те $A$,
которые лежат в круге с радиусом $\min(x_0, y_0)$ и центром в $(x_0, y_0)$.
А так как $A$ распределена равномерно,
мы можем просто поделить площадь круга на площадь прямоугольника и получить
вероятность того, что окружность из задания лежит в квадрате из задания
при условии, что центр этой окружности имеет координаты $(x_0, y_0)$:
$\frac{\pi \min(x_0, y_0)^2}{2x_0 \cdot 2y_0}$.

Тогда ответом на задачу будет
\[
    4
    \int_{0}^{\frac{1}{2}} p_0(x_0)
    \int_{0}^{\frac{1}{2}} p_0(y_0)
    \frac{\pi \min(x_0, y_0)^2}{2x_0 \cdot 2y_0}
    dy_0 dx_0
\], где $p_0(z)=4z$
(четвёрка перед интегралом нужна, потому что мы интегрируем только по четверти квадрата,
но при этом не меняем $p_0$).

\begin{equation*}
 \begin{aligned}
    4
    \int_{0}^{\frac{1}{2}} p_0(x_0)
    \int_{0}^{\frac{1}{2}} p_0(y_0)
    \frac{\pi \min(x_0, y_0)^2}{2x_0 \cdot 2y_0}
    dy_0 dx_0 &=
    4
    \int_{0}^{\frac{1}{2}} 4x_0
    \int_{0}^{\frac{1}{2}} 4y_0
    \frac{\pi \min(x_0, y_0)^2}{4 x_0 y_0}
    dy_0 dx_0=
    16 \pi
    \int_{0}^{\frac{1}{2}}\int_{0}^{\frac{1}{2}} \min(x_0, y_0)^2 dy_0 dx_0 =
    \\&=
    16 \pi \left[
    \int_{0}^{\frac{1}{2}}\int_{0}^{x_0} y_0^2 dy_0 dx_0 +
    \int_{0}^{\frac{1}{2}}\int_{x_0}^{\frac{1}{2}} x_0^2 dy_0 dx_0
    \right]=
    \\&=
    16 \pi \left[
    \int_{0}^{\frac{1}{2}}\frac{x_0^3}{3} dx_0 +
    \int_{0}^{\frac{1}{2}} x_0^2 \left(\frac{1}{2}-x_0\right) dx_0
    \right] =
    16 \pi \left[ \frac{1}{192} + \frac{1}{192} \right] = \frac{\pi}{6} \approx 0.5236
\end{aligned}
\end{equation*}

А что же результаты эксперимента?
\begin{lstlisting}
523600045 out of 1000000000 circles were contained within the square,
Thus, the result is 52.36000%
And the expected is 52.35988%
\end{lstlisting}

Замечательно.


\chapter{
Пусть имеются две независимые серии испытаний Бернулли на $n$ опытов в каждой
с вероятностью успеха $p$. $S_i$~--- количество успехов в $n$ испытаниях в $i$-ой серии.
Найти вероятность $P\left(S_1 = k\ |\ S_1 + S_2 = m\right)$.
}
Заметим, что
\[
    \begin{cases}
     k \le m \\
     m-k \le n
    \end{cases}
\]
Здесь и далее считаем, что эти условия выполняются.


\begin{equation*}
\begin{aligned}
 P\left(S_1 = k\ |\ S_1 + S_2 = m\right)
 &= \frac{P\left(S_1 = k \land S_1 + S_2 = m\right)}{P\left(S_1 + S_2 = m\right)}
 &\text{по формуле условной вероятности}
 \\&= \frac{P\left(S_1 = k \land S_2 = m - k\right)}{P\left(S_1 + S_2 = m\right)}
 &\text{немного алгебры}
 \\&= \frac{P\left(S_1 = k \right) \cdot P\left( S_2 = m - k\right)}{P\left(S_1 + S_2 = m\right)}
 &\text{независимые события}
 \\&= \frac{{n \choose k}p^k (1-p)^{n-k} \cdot {n \choose {m-k}}p^{m-k} (1-p)^{n-(m-k)} }
 {{2n \choose m}p^{m} (1-p)^{2n-m}}
 &\text{известная формула для испытаний Бернулли}
 \\&= \frac{{n \choose k} {n \choose {m-k}} p^{m} (1-p)^{2n-m}}{{2n \choose m}p^{m} (1-p)^{2n-m}}
 =\frac{{n \choose k}  {n \choose {m-k}}}{{2n \choose m}}
 &\text{ещё немного алгебры}
\end{aligned}
\end{equation*}

Сильно приятнее это выражение, к сожалению, не сделаешь.
Но зато оно хотя бы не зависит от $p$.

Давайте взглянем на часть результатов эксперимента:

\begin{lstlisting}
m = 4  k = 0  :     42 / 1092    got 03.846%, expected 04.334%
m = 4  k = 1  :    254 / 1092    got 23.260%, expected 24.768%
m = 4  k = 2  :    466 / 1092    got 42.674%, expected 41.796%
m = 4  k = 3  :    282 / 1092    got 25.824%, expected 24.768%
m = 4  k = 4  :     48 / 1092    got 04.396%, expected 04.334%
[...]
m = 13 k = 7  : 583961 / 1794151  got 32.548%, expected 32.508%
m = 13 k = 8  : 262724 / 1794151  got 14.643%, expected 14.628%
m = 13 k = 9  :  48458 / 1794151  got 02.701%, expected 02.709%
m = 13 k = 10 :   2738 / 1794151  got 00.153%, expected 00.155%
[...]
m = 19 k = 9  :   4961 / 9910    got 50.061%, expected 50.000%
m = 19 k = 10 :   4949 / 9910    got 49.939%, expected 50.000%
m = 20 k = 10 :    850 / 850     got 100.000%, expected 100.000%
\end{lstlisting}

Похоже на правду.

\chapter{
Рассмотрите схемы Бернулли при $n \in \left\{10, 100, 1000, 10000\right\}$
и $p \in \left\{0.001, 0.01, 0.1, 0.25, 0.5\right\}$.
Рассчитайте точные вероятности (где это возможно)
$P\left(S_n \in \left[\frac{n}{2} - \sqrt{npq},\frac{n}{2} + \sqrt{npq}\right]\right)$,
где $S_n$~--– количество успехов в $n$ испытаниях,
и приближенную с помощью одной из предельных теорем.
Сравните точные и приближенные вероятности.
Объясните результаты.
}

\end{document}

